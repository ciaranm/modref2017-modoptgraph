% vim: set spell spelllang=en tw=100 :

\documentclass{llncs}
\usepackage[pass]{geometry}
\usepackage{microtype}

\newcommand*\samethanks[1][\value{footnote}]{\footnotemark[#1]}

\title{Modelling and Optimisation with Graphs}

\author{
    Jessica Enright\inst{1}\thanks{This work was supported by the Engineering and Physical Sciences
    Research Council [grant numbers EP/P026842/1 and EP/M508056/1]},
    Chris Jefferson\inst{2}\samethanks[1] \and
    David Manlove\inst{3} \and
    Ciaran McCreesh\inst{3}\samethanks[1] \and
    Patrick Prosser\inst{3}\samethanks[1] \and
    Simon Rogers \inst{3} \and
    James Trimble\inst{3}\samethanks[1]
}

\institute{
    University of Stirling, Stirling, Scotland \and
    University of St Andrews, St Andrews, United Kingdom \and
    University of Glasgow, Glasgow, Scotland}

\begin{document}

\maketitle

\begin{abstract}
    Optimisation problems involving graphs are common. For example, network epidemiology
    investigates the dynamics of a disease spreading over a graph, where the vertices represent
    agents that can be infected, and edges the potentially infectious contacts between agents.  We
    are interested in modifying the graph to limit the scope of an epidemic, by deleting vertices or
    edges to limit its maximum component size.

    In kidney exchange programs such as the UK Living Kidney Sharing Schemes, patients requiring a
    kidney transplant who have a willing but incompatible donor may exchange donors with other
    similar patients. The scheme seeks to maximise the number of transplants which take place,
    subject to additional criteria; this problem can be modelled as a form of cycle packing in a
    weighted directed graph.

    A major goal in metabolomics is improving metabolite identification from mass spectrometry
    experiments. One route is through the comparison of measured fragmentation spectra with a
    database. Constraining results by shared graph substructure could lead to more specific result
    sets.

    And in computational algebra, many problems involving semigroups and monoids can be
    reinterpreted as graph homomorpism questions.

    This talk discusses the work we plan to carry out over the next three years, aimed at simplify
    and improving the modelling and solving process for such problems.  We will develop high level
    modelling support for graphs for the Essence' constraint modelling language.  Being able to
    express graph operations directly will reduce the modelling cost, and compiling from high-level
    models will simplify automatic reformulation and the use of alternative encodings.

    To solve these problems, we envision a hybrid approach, combining constraint programming or
    mixed integer solvers with dedicated subgraph algorithms. Although state-of-the-art subgraph
    algorithms often have a constraint-based feel to them, they use tailored data structures and
    propagation routines, which are critical for scalability. Propagating certain kinds of
    constraints (such as connectivity) can be done directly inside subgraph algorithms, but we
    believe a ``subgraphs modulo theories'' approach involving multiple communicating solvers is
    necessary for more complex problems. A final advantage of high level modelling, then, is in
    automating the configuration of such approaches.
\end{abstract}

\end{document}
