% vim: set spell spelllang=en tw=100 et sw=4 sts=4 foldmethod=marker foldmarker={{{,}}} :

\documentclass{beamer}

\usepackage{tikz}
\usepackage{xcolor}
\usepackage{complexity}
\usepackage{hyperref}
\usepackage{microtype}
\usepackage{amsmath}                   % \operatorname
\usepackage{amsfonts}                  % \mathcal
\usepackage{amssymb}                   % \nexists
\usepackage{gnuplot-lua-tikz}          % graphs
\usepackage[vlined]{algorithm2e} % algorithms
\usepackage{centernot}

\usetikzlibrary{shapes, arrows, shadows, calc, positioning, fit}
\usetikzlibrary{decorations.pathreplacing, decorations.pathmorphing, shapes.misc}
\usetikzlibrary{tikzmark}

\definecolor{uofguniversityblue}{rgb}{0, 0.219608, 0.396078}

\definecolor{uofgheather}{rgb}{0.356863, 0.32549, 0.490196}
\definecolor{uofgaquamarine}{rgb}{0.603922, 0.72549, 0.678431}
\definecolor{uofgslate}{rgb}{0.309804, 0.34902, 0.380392}
\definecolor{uofgrose}{rgb}{0.823529, 0.470588, 0.709804}
\definecolor{uofgmocha}{rgb}{0.709804, 0.564706, 0.47451}
\definecolor{uofgsandstone}{rgb}{0.321569, 0.278431, 0.231373}
\definecolor{uofgforest}{rgb}{0, 0.2, 0.129412}
\definecolor{uofglawn}{rgb}{0.517647, 0.741176, 0}
\definecolor{uofgcobalt}{rgb}{0, 0.615686, 0.92549}
\definecolor{uofgturquoise}{rgb}{0, 0.709804, 0.819608}
\definecolor{uofgsunshine}{rgb}{1.0, 0.862745, 0.211765}
\definecolor{uofgpumpkin}{rgb}{1.0, 0.72549, 0.282353}
\definecolor{uofgthistle}{rgb}{0.584314, 0.070588, 0.447059}
\definecolor{uofgrust}{rgb}{0.603922, 0.227451, 0.023529}
\definecolor{uofgburgundy}{rgb}{0.490196, 0.133333, 0.223529}
\definecolor{uofgpillarbox}{rgb}{0.701961, 0.047059, 0}
\definecolor{uofglavendar}{rgb}{0.356863, 0.301961, 0.580392}

\tikzset{vertex/.style={draw, circle, inner sep=0pt, minimum size=0.5cm, font=\small\bfseries}}
\tikzset{notvertex/.style={vertex, color=white, text=black}}
\tikzset{plainvertex/.style={vertex}}
\tikzset{vertexc1/.style={vertex, fill=uofgburgundy, text=white}}
\tikzset{vertexc2/.style={vertex, fill=uofgsandstone, text=white}}
\tikzset{vertexc3/.style={vertex, fill=uofgforest, text=white}}
\tikzset{vertexc4/.style={vertex, fill=uofgheather, text=white}}
\tikzset{edge/.style={color=black!50!white}}
\tikzset{bedge/.style={ultra thick}}

% {{{ theme things
\useoutertheme[footline=authortitle]{miniframes}
\useinnertheme{rectangles}

\setbeamerfont{block title}{size={}}
\setbeamerfont{title}{size=\large,series=\bfseries}
\setbeamerfont{section title}{size=\large,series=\mdseries}
\setbeamerfont{author}{size=\normalsize,series=\mdseries}
\setbeamercolor*{structure}{fg=uofguniversityblue}
\setbeamercolor*{palette primary}{use=structure,fg=black,bg=white}
\setbeamercolor*{palette secondary}{use=structure,fg=white,bg=uofgcobalt}
\setbeamercolor*{palette tertiary}{use=structure,fg=white,bg=uofguniversityblue}
\setbeamercolor*{palette quaternary}{fg=white,bg=black}

\setbeamercolor*{titlelike}{parent=palette primary}

\beamertemplatenavigationsymbolsempty

\setbeamertemplate{title page}
{
    \begin{tikzpicture}[remember picture, overlay]
        \node at (current page.north west) {
            \begin{tikzpicture}[remember picture, overlay]
                \fill [fill=white, anchor=north west, yshift=-0.15cm, opacity=0.7] (0, 0) rectangle (\paperwidth, -3.2cm);
            \end{tikzpicture}
        };

        \node at (current page.north west) {
            \begin{tikzpicture}[remember picture, overlay]
                \fill [fill=white, anchor=north west, yshift=-0.15cm, xshift=0.7\paperwidth] (0, 0) rectangle (0.3\paperwidth, -3.2cm);
            \end{tikzpicture}
        };

        \node (logo) [anchor=north east, shift={(-0.15cm,-0.20cm)}] at (current page.north east) {
            \begin{minipage}{0.25\paperwidth}
                \begin{center}
                    \includegraphics*[keepaspectratio=true,scale=0.45]{UoG_colour.pdf} \\[-0.05cm]
                    \includegraphics*[keepaspectratio=true,scale=1.00]{UStA_Logo_SP.pdf} \\[-0.05cm]
                    \includegraphics*[keepaspectratio=true,scale=0.11]{UoS-LOGO-PMS-349.pdf}
                \end{center}
            \end{minipage}
        };

        \node [anchor=west, xshift=0.1cm] at (current page.west |- logo.west) {
            \begin{minipage}{0.67\paperwidth}\raggedright
                {\usebeamerfont{title}\usebeamercolor[black]{}\inserttitle}\\[0.2cm]
                {\usebeamerfont{author}\usebeamercolor[black]{}\insertauthor}
            \end{minipage}
        };
    \end{tikzpicture}
}

\setbeamertemplate{section page}
{
    \begin{centering}
        \begin{beamercolorbox}[sep=12pt,center]{part title}
            \usebeamerfont{section title}\insertsection\par
        \end{beamercolorbox}
    \end{centering}
}

\newcommand{\frameofframes}{/}
\newcommand{\setframeofframes}[1]{\renewcommand{\frameofframes}{#1}}

\makeatletter
\setbeamertemplate{footline}
{%
    \begin{beamercolorbox}[colsep=1.5pt]{upper separation line foot}
    \end{beamercolorbox}
    \begin{beamercolorbox}[ht=2.5ex,dp=1.125ex,%
        leftskip=.3cm,rightskip=.3cm plus1fil]{author in head/foot}%
        \leavevmode{\usebeamerfont{author in head/foot}\insertshortauthor}%
        \hfill%
        {\usebeamerfont{institute in head/foot}\usebeamercolor[fg]{institute in head/foot}\insertshortinstitute}%
    \end{beamercolorbox}%
    \begin{beamercolorbox}[ht=2.5ex,dp=1.125ex,%
        leftskip=.3cm,rightskip=.3cm plus1fil]{title in head/foot}%
        {\usebeamerfont{title in head/foot}\insertshorttitle}%
        \hfill%
        {\usebeamerfont{frame number}\usebeamercolor[fg]{frame number}\insertframenumber~\frameofframes~\inserttotalframenumber}
    \end{beamercolorbox}%
    \begin{beamercolorbox}[colsep=1.5pt]{lower separation line foot}
    \end{beamercolorbox}
}

\makeatletter
\newenvironment{nearlyplainframe}[2][]{
    \def\beamer@entrycode{\vspace*{-\headheight}\vspace*{3pt}}
    \setbeamertemplate{headline}
    {%
        \begin{beamercolorbox}[colsep=1.5pt]{upper separation line head}
        \end{beamercolorbox}
        \begin{beamercolorbox}[ht=0.5ex,dp=0.125ex,%
            leftskip=.3cm,rightskip=.3cm plus1fil]{title in head/foot}%
        \end{beamercolorbox}%
        \begin{beamercolorbox}[ht=0.5ex,dp=0.125ex,%
            leftskip=.3cm,rightskip=.3cm plus1fil]{author in head/foot}%
        \end{beamercolorbox}%
        \begin{beamercolorbox}[colsep=1.5pt]{lower separation line head}
        \end{beamercolorbox}
        \vspace*{\headheight}
    }

    \setbeamertemplate{footline}
    {%
        \begin{beamercolorbox}[colsep=1.5pt]{upper separation line foot}
        \end{beamercolorbox}
        \begin{beamercolorbox}[ht=0.5ex,dp=0.125ex,%
            leftskip=.3cm,rightskip=.3cm plus1fil]{author in head/foot}%
        \end{beamercolorbox}%
        \begin{beamercolorbox}[ht=0.5ex,dp=0.125ex,%
            leftskip=.3cm,rightskip=.3cm plus1fil]{title in head/foot}%
        \end{beamercolorbox}%
        \begin{beamercolorbox}[colsep=1.5pt]{lower separation line foot}
        \end{beamercolorbox}
    }

    \begin{frame}[#1]{#2}
    }{
    \end{frame}
}
\makeatother

% }}}

\author[Jessica Enright \and Chris Jefferson \and David Manlove \and Ciaran McCreesh \and Patrick Prosser \and Simon Rogers \and James Trimble]{
    Jessica Enright,
    Chris Jefferson,
    David Manlove,
    \textbf{Ciaran McCreesh},
    Patrick Prosser,
    Simon Rogers,
    James Trimble,
    and \textbf{Hiring Another Postdoc!}}

\title[Modelling and Optimisation with Graphs]{Modelling and Optimisation with Graphs}

\begin{document}

{
    \usebackgroundtemplate{
        \tikz[overlay, remember picture]
        \node[at=(current page.south), anchor=south, inner sep=0pt]{\includegraphics*[keepaspectratio=true, height=\paperheight]{background.jpg}};
    }
    \begin{frame}[plain,noframenumbering]
        \titlepage
    \end{frame}
}

\section{Application Case Studies}

\begin{frame}{Motivation}
    \begin{itemize}
        \item Lots of interesting, practical, important optimisation problems involve graphs and
            subgraphs, together with side constraints which cannot easily be expressed in graph
            terms.
    \end{itemize}
\end{frame}

\begin{frame}{Kidney Exchange}
    \only<1>{
        \centering
        \includegraphics*[keepaspectratio=true,scale=0.25]{kidney1.png}
    }
    \only<2>{
        \centering
        \includegraphics*[keepaspectratio=true,scale=0.25]{kidney2.png}
    }
    \only<3>{
        \centering
        \includegraphics*[keepaspectratio=true,scale=0.35]{kidney4.png}
    }
    \only<4>{
        \centering
        \includegraphics*[keepaspectratio=true,scale=0.25]{kidney3.png}
    }
\end{frame}

\begin{frame}{Network Epidemiology (Diseased Cows)}
    \only<1>{
        \centering
        \includegraphics*[keepaspectratio=true,scale=0.25]{cows3.png}
    }
    \only<2>{
        \centering
        \includegraphics*[keepaspectratio=true,scale=0.20]{cows1.png}
    }
    \only<3>{
        \centering
        \includegraphics*[keepaspectratio=true,scale=0.25]{cows2.png}
    }
    \only<4>{
        \begin{itemize}
            \item Delete as few edges as possible from a graph until the maximum component size is
                less than $k$.
            \item There's a really elegant CP model for this.
            \item But a dedicated treewidth-based algorithm works better in practice\ldots
        \end{itemize}
    }
\end{frame}

\begin{frame}{Metabolomics}
    \only<1>{
        \centering
        \includegraphics*[keepaspectratio=true,scale=0.45]{metabolomics1.png}
    }\only<2>{
        \centering
        \includegraphics*[keepaspectratio=true,scale=0.45]{metabolomics2.png}
    }
\end{frame}

\begin{frame}{Computational Algebra}
    \only<1>{
        \centering
        \includegraphics*[keepaspectratio=true,scale=0.25]{ca1.png}
    }
\end{frame}

\section{High Level Modelling}

\begin{frame}{Essence'}
    \begin{itemize}
        \item Essence' is a high level modelling language developed at St~Andrews.
        \item Objectively over nine thousand times better than MiniZinc.
    \end{itemize}
\end{frame}

\begin{frame}{Expressiveness}
    \begin{itemize}
        \item We'd like to work with graphs as high level types.
    \end{itemize}
\end{frame}

\begin{frame}{Compilation and Automatic Reformulation}
    \begin{itemize}
        \item There is more than one way to represent a graph.
        \item Kidney exchange in particular has some very clever modelling techniques which we'd
            like to reuse for other problems.
        \item We do \emph{not} envision graph variables within CP / MIP solvers.
    \end{itemize}
\end{frame}

\section{Solver Support}

\begin{frame}{Subgraph Modulo Theories}
    \begin{itemize}
        \item State of the art subgraph solvers have a strong CP feel to them, but with different
            data structures, propagation techniques, and algorithms.

        \item We think it might be possible to share nogoods between solvers.

        \item High level modelling might make this less of a pain to set up.
    \end{itemize}
\end{frame}

\begin{frame}{Microstructure / Weighted Clique}
    \begin{itemize}
        \item Pet project: reducing certain constraint optimisation problems to maximum weight
            clique, via a microstructure-like encoding.
        \item CP last year: using an unweighted clique solver is actually a really good idea for
            maximum common subgraph.
            \begin{itemize}
                \item We can also propagate certain side constraints inside a clique algorithm.
            \end{itemize}
        \item CP this year: weighted clique solvers are terrible because they've been benchmarked on
            bad instances.
        \item CP next year: we have a new weighted clique solver and I was right the whole time and
            I'm not crazy!
    \end{itemize}
\end{frame}

\section{Hiring a Postdoc! Google AR1972SB}

\begin{frame}{We're Hiring!}
    \centering
    \includegraphics*[keepaspectratio=true,scale=0.32]{standrews.jpg} \\
    \href{https://www.vacancies.st-andrews.ac.uk/}{\nolinkurl{https://www.vacancies.st-andrews.ac.uk/}}
    \\ reference AR1972SB
\end{frame}

\begin{frame}[plain,noframenumbering]
    \begin{center}
        \vspace*{4em}
        \url{http://www.dcs.gla.ac.uk/~ciaran} \\
        \vspace{1em}
        \href{mailto:ciaran.mccreesh@glasgow.ac.uk}{\nolinkurl{ciaran.mccreesh@glasgow.ac.uk}} \\
        \vspace{1em}
        \href{https://www.vacancies.st-andrews.ac.uk/}{\nolinkurl{https://www.vacancies.st-andrews.ac.uk/}}
        \\ reference AR1972SB
    \end{center}

    \begin{tikzpicture}[remember picture, overlay]
        \node at (current page.north west) {
            \begin{tikzpicture}[remember picture, overlay]
                \fill [fill=uofguniversityblue, anchor=north west] (0, 0) rectangle (\paperwidth, -3.2cm);
            \end{tikzpicture}
        };
    \end{tikzpicture}
\end{frame}

\end{document}

